\section{Air Velocity}
Did you ever feel the wind blow? Most probably. That's what we will be calculating here. How hard the wind will blow. This is noted as velocity, how fast something moves. 

\subsection{Equation of State and the Incompressible Atmosphere}
The equation of state relates one or more variables in a dynamical system (like the atmosphere) to another. The most common equation of state in the atmosphere is the ideal gas equation as 
described by \autoref{eq:ideal gas} \cite{idealGas}. The symbols in that equation represent:

\begin{itemize}
    \item $p$: The gas pressure (\si{Pa}).
    \item $V$: The volume of the gas (\si{m^3}).
    \item $n$: The amount of moles in the gas (\si{mol}).
    \item $R$: The Gas constant as defined in \autoref{sec:gas constant} (\si{JK^{-1}mol^{-1}}) \cite{idealGas}.
    \item $T$: The temperature opf the gas ($K$).
\end{itemize}

If we divide everything in \autoref{eq:ideal gas} by $V$ and set it to be unit (in this case, set it to be exactly $1$ \si{m^3}) we can add in the molar mass in both the top and bottom parts of 
the division as show in \autoref{eq:gas unit}. We can then replace $\frac{nm}{V}$ by $\rho$ the density of the gas (\si{kgm^{-3}}) and $\frac{R}{m}$ by $R_s$ the specific gas constant (gas 
constant that varies per gas in \si{JK^{-1}mol^{-1}}) as shown in \autoref{eq:state gas}. The resulting equation is the equation of state that you get that most atmospheric physicists use when 
talking about the atmosphere \cite{simon}.

\begin{subequations}
    \begin{equation}
        \label{eq:ideal gas}
        pV = nRT
    \end{equation}
    \begin{equation}
        \label{eq:gas unit}
        p = \frac{nR}{V}T = \frac{nmR}{Vm}T
    \end{equation}
    \begin{equation}
        \label{eq:state gas}
        p = \rho R_sT
    \end{equation}
\end{subequations}

The pressure is quite important, as air moves from a high pressure point to a low pressure point. So if we know the density and the temperature, then we know the pressure and we can work out 
where the air will be moving to (i.e. how the wind will blow). In our current model, we know the atmospheric temperature but we do not know the density. For simplicities sake, we will now assume
that the atmosphere is Incompressible, meaning that we have a constant density. Obviously we know that air can be compressed and hence our atmosphere can be compressed too but that is not 
important enough to account for yet, especially considering the current complexity of our model.

The code that corresponds to this is quite simple, the only change that we need to make in \autoref{eq:state gas} is that we need to replace $T$ by $T_a$, the temperature of the atmosphere. As
$T_a$ is a matrix (known to programmers as a double array), $p$ will be a matrix as well. Now we only need to fill in some values. $\rho = 1.2$\cite{densityAir}, $R_s = 287$\cite{specificGasConstantAir}.

\subsection{The Momentum Equations} \label{sec:momentum}
The momentum equations are a set of equations that describe the flow of a fluid on the surface of a rotating body. For our model we will use the f-plane approximation. The equations corresponding
to the f-plane approximation are given in \autoref{eq:x momentum} and \autoref{eq:y momentum} \cite{momentumeqs}. Note that we are ignoring vertical movement, as this does not have a significant
effect on the whole flow. All the symbols in \autoref{eq:x momentum} and \autoref{eq:y momentum} mean:

\begin{itemize}
    \item $u$: The east to west velocity (\si{ms^{-1}}).
    \item $t$: The time (\si{s}).
    \item $f$: The coriolis parameter as in \autoref{eq:coriolis}.
    \item $v$: The north to south velocity (\si{ms^{-1}}).
    \item $\rho$: The density of the atmosphere (\si{kgm^{-3}}).
    \item $p$: The atmospheric pressure (\si{Pa}).
    \item $x$: The local longitude coordinate (\si{m}).
    \item $y$: The local latitude coordinate (\si{m}).
\end{itemize}

If we then define a vector $\bar{u}$ as $(u, v, 0)$, we can rewrite both \autoref{eq:x momentum} as \autoref{eq:x momentum laplace}. Here $\nabla u$ is the gradient of $u$ in both $x$ and $y$ 
directions. Then if we write out $\nabla u$ we get \autoref{eq:x momentum final}. Similarly, if we want to get $\delta v$ instead of $\delta u$ we rewrite \autoref{eq:y momentum} to get 
\autoref{eq:y momentum laplace} and \autoref{eq:y momentum final}.

\begin{subequations}
    \begin{equation}
        \label{eq:x momentum}
        \frac{Du}{Dt} - fv = -\frac{1}{\rho} \frac{\delta p}{\delta x}
    \end{equation}
    \begin{equation}
        \label{eq:y momentum}
        \frac{Dv}{Dt} - fu = -\frac{1}{\rho} \frac{\delta p}{\delta y}
    \end{equation}
    \begin{equation}
        \label{eq:x momentum laplace}
        \frac{\delta u}{\delta t} + \bar{u} \cdot \nabla u - fv = -\frac{1}{\rho}\frac{\delta p}{\delta x}
    \end{equation}
    \begin{equation}
        \label{eq:y momentum laplace}
        \frac{\delta v}{\delta t} + \bar{u} \cdot \nabla v - fu = -\frac{1}{\rho}\frac{\delta p}{\delta y}
    \end{equation}
    \begin{equation}
        \label{eq:x momentum final}
        \frac{\delta u}{\delta t} + u\frac{\delta u}{\delta x} + v\frac{\delta u}{\delta y} - fv = -\frac{1}{\rho}\frac{\delta p}{\delta x}
    \end{equation}
    \begin{equation}
        \label{eq:y momentum final}
        \frac{\delta v}{\delta t} + u\frac{\delta v}{\delta x} + v\frac{\delta v}{\delta y} - fu = -\frac{1}{\rho}\frac{\delta p}{\delta y}
    \end{equation}
\end{subequations}

With the gradient functions defined in \autoref{alg:gradient x} and \autoref{alg:gradient y}, we can move on to the main code for the momentum equations. The main loop is shown in 
\autoref{alg:stream3}. Do note that this loop replaces the one in \autoref{alg:stream2v2} as these calculate the same thing, but the new algorithm does it better.

\begin{algorithm}
    \caption{Calculating the flow of the atmosphere (wind)}
    \label{alg:stream3}
    $S_{xu} \leftarrow \texttt{gradient\_x}(u, lat, lon)$ \;
    $S_{yu} \leftarrow \texttt{gradient\_y}(u, lat, lon)$ \;
    $S_{xv} \leftarrow \texttt{gradient\_x}(v, lat, lon)$ \;
    $S_{yv} \leftarrow \texttt{gradient\_y}(v, lat, lon)$ \;
    $S_{px} \leftarrow \texttt{gradient\_x}(p, lat, lon)$ \;
    $S_{py} \leftarrow \texttt{gradient\_x}(p, lat, lon)$ \;
    \For{$lat \leftarrow 1$ \KwTo $nlat - 1$}{
        \For{$lon \leftarrow 0$ \KwTo $nlon$}{
            $u[lat, lon] \leftarrow u[lat, lon] + \delta t \frac{-u[lat, lon]S_{xu} - v[lat, lon]S_{yu} + f[lat]v[lat, lon] - S_{px}}{\rho}$ \;
            $v[lat, lon] \leftarrow v[lat, lon] + \delta t \frac{-u[lat, lon]S_{xv} - v[lat, lon]S_{yv} - f[lat]u[lat, lon] - S_{py}}{\rho}$ \;
        }
    }
\end{algorithm}

\subsection{Improving the Coriolis Parameter}
Another change introduced is in the coriolis parameter. Up until now it has been a constant, however we know that it varies along the latitude. So let's make it vary over the latitude. Recall 
\autoref{eq:coriolis}, where $\Theta$ is the latitude. Coriolis ($f$) is currently defined in \autoref{alg:gradient}, so let's replace it with \autoref{alg:coriolis}.

\begin{algorithm}
    \caption{Calculating the coriolis force}
    \label{alg:coriolis}
    \SetAlgoLined
    $\Omega \leftarrow 7.2921 \cdot 10^{-5}$ \;

    \For{$lat \leftarrow -nlat$ \KwTo $nlat$}{
        $f[lat] \leftarrow 2\Omega \sin(lat \frac{\pi}{180})$ \;
    }
\end{algorithm}

\subsection{Adding Friction}
In order to simulate friction, we multiply the speeds $u$ and $v$ by $0.99$. Of course there are equations for friction but that gets complicated very fast, so instead we just assume that we
have a constant friction factor. This multiplication is done directly after \autoref{alg:stream3} in \autoref{alg:stream4v1}.

\subsection{Adding in Layers}
With adding in atmospheric layers we need to add vertical winds, or in other words add the $w$ component of the velocity vectors. We do that by editing \autoref{alg:stream3}. We change it to 
\autoref{alg:velocity}. Here we use gravity ($g$) instead of the coriolis force ($f$) and calculate the change in pressure. Therefore we need to store a copy of the pressure before we do any 
calculations. This needs to be a copy due to aliasing \footnote{Aliasing is assigning a different name to a variable, while it remains the same variable. Take for instance that we declare a 
variable $x$ and set it to be $4$. Then we say $y \leftarrow x$, which you might think is the same as saying they $y \leftarrow 4$ but behind the screen it is pointing to $x$. So if $x$ changes, 
then so does $y$.}. Since we use pressure as the vertical coordinate, we must be able to convert that into meters (why we opted for pressure is explained in \autoref{sec:rad layers}) in order to 
be able to say something sensible about it. To do that we need the concept of geopotential height.

\subsubsection{Dimensionless Pressure}
Geopotential height is similar to geometric height, except that it also accounts for the variation in gravity over the planet \cite{geopot}. One could say that geopotential height is the 
"gravity adjusted" height. That means that it is similar to the height, but not exactly the same. Height is a linear function, whereas the geopotential height is not, though it is very similar 
to a linear function if you would plot it. Now to convert easily to and from potential temperature into temperature, we need another function which is known as the Exner function. The Exner 
function is a dimensionless \footnote{Being dimensionless means that there is no dimension (unit) attached to the number. This is useful for many applications and is even used in daily life. For 
instance when comparing price rises of different products, it is way more useful to talk about percentages (who are unitless) instead of how much you physically pay more (with your favourite 
currency as the unit).} pressure. The Exner function is shown in \autoref{eq:exner} \cite{verticalcoords}. The symbols in the equation are: 

\begin{itemize}
    \item $c_p$: The specific heat capacity of the atmosphere.
    \item $p$: Pressure (\si{Pa}).
    \item $p_0$: Reference pressure to define the potential temperature (\si{Pa}).
    \item $R$: The gas constant $8.3144621$ (\si{J(mol)^{-1}K}).
    \item $T$: The absolute temperature (\si{K}).
    \item $\theta$: the potential temperature (\si{K}).
\end{itemize}

Since the right hand side contains what we want to convert to and from, we can do some basic rewriting, which tells us what we need to code to convert potential temperature in absolute 
temperature and vice versa. This is shown in \autoref{eq:temp exner} and \autoref{eq:potential temp exner} respectively.

\begin{subequations}
    \begin{equation}
        \label{eq:exner}
        \Pi = c_p(\frac{p}{p_0})^{\frac{R}{c_p}} = \frac{T}{\theta}
    \end{equation}
    \begin{equation}
        \label{eq:temp exner}
        T = \Pi\theta
    \end{equation}
    \begin{equation}
        \label{eq:potential temp exner}
        \theta = \frac{T}{\Pi}
    \end{equation}
\end{subequations}

Now onto some code. Let us initialise $\Pi$ before we do any other calculations. This code is already present in the control panel section (\autoref{sec:cp}) as that is where it belongs, so for 
further details please have a look at the code there. Now onto the geopotential height. 

\subsubsection{Geopotential Height}
As stated before, geopotential height is similar to geometric height, except that it also accounts for the variation in gravity over the planet. One could say that geopotential height is the 
"gravity adjusted" height. That means that it is similar to the height, but not exactly the same. Height is a linear function, whereas the geopotential height is not, though it is very similar 
to a linear function if you would plot it. Now one could ask why we would discuss dimensionless pressure before geopotential height. The answer is quite simple, in order to define geopotential 
height, we need the Exner function to define it. Or rather, we need that function to convert potential temperature into geopotential height. How those three are related is shown in 
\autoref{eq:geopot}. Then with a little transformation we can define how the geopotential height will look like, as shown in \autoref{eq:geopot final}. The symbols in both equations are:

\begin{itemize}
    \item $\Pi$: The Exner function.
    \item $\theta$: Potential temperature (\si{K}).
\end{itemize}

\begin{subequations}
    \begin{equation}
        \label{eq:geopot}
        \theta + \frac{\delta\Phi}{\delta\Pi} = 0
    \end{equation}
    \begin{equation}
        \label{eq:geopot final}
        \delta\Phi = -\theta\delta\Pi
    \end{equation}
\end{subequations}

Now to turn this into code we need to be careful about a few things. First we are talking about a change in geopotential height here, so defining one level of geopotential height means that it 
is dependent on the level below it. Second this calculatoin needs potential temperature and therefore it should be passed along to the velocity calculations function. With those two things out 
of the way, we get the code as shown in \autoref{alg:geopot}. Note that \texttt{Smooth3D} refers to \autoref{alg:smooth}.

\begin{algorithm}
    \caption{Calculating the geopotential height}
    \label{alg:geopot}
    \For{$level \leftarrow 1$ \KwTo $nlevels$}{
        $\Phi[:, :, level] \leftarrow \Phi[:, :, level - 1] - T_{pot}[:, :, level](\Pi[level] - \Pi[level - 1])$ \;
    }
    $\Phi \leftarrow \texttt{Smooth3D}(\Phi, smooth_t)$ \;
\end{algorithm}

\subsubsection{Finally Adding in the Layers}
Now with the geopotential height and dimensionless pressure out of the way, we need to use those two concepts to add layers to the velocity calculations. Before we dive into the code however, 
there are slight changes that we need to discuss. The equation shown in \autoref{eq:velocity} is the primitive equation (as discussed in \autoref{sec:primitive}). The momentum equations are 
gesostrphic momentum, which are a special form of the primitive equation. Since this whole system must remain in equilibrium, we need to set the right hand side to $0$ as shown in 
\autoref{eq:vel eq}. Now let us rewrite \autoref{eq:velocity} into \autoref{eq:velocity int}. We replaze $z$ with pressure as that is our vertical coordinate. $\omega$ is the velocity of the 
pressure field, as defined in \autoref{eq:vert vel}. Note that $p_k$ is the pressure for layer $k$ and $p_0$ is the pressure at the planet surface. Now we need to turn the velocity of the 
pressure field into the velocity of a packet of air (see it as a box of air being moved), which is done in \autoref{eq:vertical velocity}. Here $\rho$ is the density and $g$ is gravity 
(\si{ms^{-2}}).

\begin{subequations}
    \begin{equation}
        \label{eq:velocity}
        \frac{\delta T}{\delta x} + \frac{\delta T}{ \delta y} + \frac{\delta T}{\delta z} = \nabla T
    \end{equation}
    \begin{equation}
        \label{eq:vel eq}
        \nabla T = 0
    \end{equation}
    \begin{equation}
        \label{eq:velocity int}
        \frac{\delta u}{\delta x} + \frac{\delta v}{\delta y} + \frac{\delta\omega}{\delta p} = 0
    \end{equation}
    \begin{equation}
        \label{eq:vert vel}
        \omega_k = -\int^{p_k}_{p_0}\frac{\delta u}{\delta x} + \frac{\delta v}{\delta y} dp
    \end{equation}
    \begin{equation}
        \label{eq:vertical velocity}
        w = \omega \rho g
    \end{equation}
\end{subequations}

But I hear you say, what is the density? Since we have moved to pressure coordinates we can actually calculate the density rather than store it. This is done in \autoref{eq:density}, where each 
symbol means:

\begin{itemize}
    \item $\rho$: The density of the atmosphere.
    \item $p$: The pressure of the atmosphere (\si{Pa}).
    \item $c$: Specific heat capacity of the atmosphere (\si{JKg^{-1}K^{-1}}).
    \item $T$: Temperature of the atmosphere (\si{K}).
\end{itemize}

\begin{equation}
    \label{eq:density}
    \rho = \frac{p}{cT}
\end{equation}

Finally, let us convert \autoref{eq:vertical velocity} to code in \autoref{alg:velocity}. Here $T_{trans}$ is a call to the algorithm as described in \autoref{alg:temp to pot}. 
\texttt{gradient\_x}, \texttt{gradient\_y} and \texttt{gradient\_z} are calls to \autoref{alg:gradient x}, \autoref{alg:gradient y} and \autoref{alg:gradient z} respectively.

\begin{algorithm}
    \caption{Calculating the flow of the atmosphere (wind)}
    \label{alg:velocity}
    //\texttt{The following variables are function calls to algorithms and their shorthand notations are used in the loops}\\
    $S_{xu} \leftarrow \texttt{gradient\_x}(u, lat, lon, layer)$ \;
    $S_{yu} \leftarrow \texttt{gradient\_y}(u, lat, lon, layer)$ \;
    $S_{xv} \leftarrow \texttt{gradient\_x}(v, lat, lon, layer)$ \;
    $S_{yv} \leftarrow \texttt{gradient\_y}(v, lat, lon, layer)$ \;
    $S_{zu} \leftarrow \texttt{gradient\_z}(u[lat, lon], p_z, layer)$ \;
    $S_{zv} \leftarrow \texttt{gradient\_z}(v[lat, lon], p_z, layer)$ \;
    $S_{px} \leftarrow \texttt{gradient\_x}(geopot, lat, lon, layer)$ \;
    $S_{py} \leftarrow \texttt{gradient\_y}(geopot, lat, lon, layer)$ \;

    //\texttt{The following variables are real variables}\\
    $nlat \leftarrow \Phi.length$ \;
    $nlon \leftarrow \Phi[0].length$ \;
    $u_t \leftarrow $ array like $u$ \;
    $v_t \leftarrow $ array like $v$ \;
    $w_t \leftarrow $ array like $w$ \;
    \For{$lat \leftarrow 1$ \KwTo $nlat - 1$}{
        \For{$lon \leftarrow 0$ \KwTo $nlon$}{
            \For{$layer \leftarrow 0$ \KwTo $nlevels$}{
                $u_t[lat, lon, layer] \leftarrow u[lat, lon, layer] + \delta t \frac{-u[lat, lon, layer]S_{xu} - v[lat, lon, layer]S_{yu} + f[lat]v[lat, lon, layer] - S_{px}}
                    {10^5u[lat, lon, layer]}$ \;
                $v_t[lat, lon, layer] \leftarrow v[lat, lon, layer] + \delta t \frac{-u[lat, lon, layer]S_{xv} - v[lat, lon, layer]S_{yv} - f[lat]u[lat, lon, layer] - S_{py}}
                    {10^5v[lat, lon, layer]}$ \;
            }
        }
    }

    $T_a \leftarrow T_{trans}(T_{pot}, p_z, \texttt{False})$ \;

    \For{$lat \leftarrow 2$ \KwTo $nlat - 2$}{
        \For{$lon \leftarrow 0$ \KwTo $nlon$}{
            \For{$level \leftarrow 1$ \KwTo $nlevels$}{
                $w_t[lat, lon, level] \leftarrow  w_t[lat, lon, level - 1] - \frac{(p_z[level] - p_z[level - 1])p_z[level]g(S_{xu} + S_{yv})}{C_pT_a[lat, lon, layer]}$ \;
            }
        }
    }

    $u \leftarrow u + u_t$ \;
    $v \leftarrow v + v_t$ \;
    $w \leftarrow w + w_t$ \;
    $p_0 \leftarrow copy(p)$ \;
\end{algorithm}